\documentclass[11pt]{article}

\usepackage[utf8x]{inputenc}		%Input Encoding
\usepackage[T1]{fontenc}				%Font Encoding
\usepackage[french]{babel}		%Language
\usepackage[top=1in, bottom=1.25in, left=1.25in, right=1.25in]{geometry}
\usepackage{lmodern}					%Font
\usepackage{parskip}						%Space Between Paragraphs
\usepackage{amsmath}				%
\usepackage{amssymb}				%
\usepackage{mathrsfs}				%
\usepackage{graphicx}				%
\usepackage{textcomp}				% 
\usepackage{array}
\usepackage{siunitx}				%International System of Units
\usepackage{ marvosym }
\usepackage{titlesec, blindtext, color}			%sections format
\usepackage{hyperref}
\usepackage{url}

\definecolor{gray75}{gray}{0.75}
\newcommand{\hsp}{\hspace{20pt}}
\titlespacing*{\part}{0pt}{0pt}{20pt}
\titleformat{\part}[block]{\LARGE\bfseries}{\thepart\hsp\textcolor{gray75}{}\hsp}{0pt}{\LARGE\bfseries}
\titlespacing*{\part}{0pt}{0pt}{20pt}
\titleformat{\part}[block]{\LARGE\bfseries}{\thepart\hsp\textcolor{gray75}{|}\hsp}{0pt}{\LARGE\bfseries}
\titleformat{\section}[block]{\normalfont\bfseries}	{\thesection}					{1ex}{\Large}		[]


\begin{document}

\setlength{\parskip}{10pt} % 1ex plus 0.5ex minus 0.2ex}
\setlength{\parindent}{20pt}
\pagestyle{plain}

%title
\begin{titlepage}
	\centering
	\includegraphics[scale=0.4]{epl-logo.jpg}\\
	\vspace{2cm}	
	{\scshape\Large LFSAB1402: Rapport de Projet\par}
	\vspace{2cm}
	{\huge\bfseries Projet BomberSnake Oz \par}
%%	\vfill
%%	{\Large\itshape Groupe 11.52\par}
	\vspace{0.5cm}
	{\large Audrick Deckers --- 8386--16--00\par Laurent Ziegler de Ziegleck a\`{u}f Rheingr\"{u}b --- 0382--15--00\par}
	\vfill

% Bottom of the page
	{\large \today \par Année Académique 2017--2018\par}
\end{titlepage}

%\tableofcontents
\newpage

\part{Introduction}
Suite à l'intéret un peu trop prononcé de certains étudiants pour le jeu "Snake" implémenté dans la console Emacs grâce à laquelle nous pouvons coder en Oz, les asssitants du cours \emph{LFSAB -- Informatique 2} nous ont demandé de recoder un "Snake" en Oz.

Pour ce faire, il nous a été fourni la base du code et nous a été demandé d'implémenter les fonctions \emph{Next} et \emph{DecodeInstruction} ainsi trois effets imposés (\emph{Grow}, \emph{Revert} et \emph{Teleport}) et deux au choix (\emph{Casa} et \emph{Jump} dans notre cas).


\part{Choix de Conception}



\part{Analyse de Complexité}



\part{Description des Extensions}



\part{Conclusion}




\end{document}
